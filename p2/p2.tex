\documentclass[../main.tex]{subfiles}

\begin{document}
\begin{enumerate}[a)]
    \item Para calcular la razón de mezcla usamos la humedad relativa y la relación
        \begin{equation}
            w = \frac{\text{HR}}{100} w_\text{sat}  
        \end{equation}
Sin embargo para poder usar esta expresión necesitamos también $w_\text{sat}$. Esta cantidad lo obtenemos a partir de
\begin{equation}
    w_\text{sat} = e_\text{sat} \frac{\epsilon}{P - e_\text{sat}}
\end{equation}
y por último, $e_\text{sat}$ la obtenemos usando la ecuación de Clasius-Clapeyron, en particular la siguiente forma de dicha ecuación:
\begin{equation}
   e_\text{sat}(T) 6.11 \cdot \exp \left(   5.42\cdot 10^3 \left( \frac{1}{273} - \frac{1}{T} \right) \right)
\end{equation}
con $T$ la temperatura en Kelvin, y $\epsilon = 0.622$.\\

A continuación mostramos los valores obtenidos para la razón de mezcla de cada nivel.\\

\begin{minipage}{\linewidth}
\centering
\captionof{table}{Datos de presión, temperatura, humedad relativa, presión de vapor saturada, razón de mezcla saturada, y razón de mezcla. Todo esto en los 5 niveles mas bajos.}
\import{data}{w.tex}
\end{minipage}\\

Para calcular la cantidad de agua usamos la siguiente expresión
\begin{equation}
    LWC = \frac{1}{\rho_l g} \int_{P_1}^{P_2} q(p) dp
,\end{equation}
con $\rho_l$ la densidad del agua líquida, $g$ la aceleración de gravedad, $q$ la humedad específica y $p$ la presión. 

Como en realidad tenemos una cantidad discreta y no podemos integrar, haremos la siguiente suma. 
 \begin{equation}
     LWC = \frac{1}{\rho_l g} \sum \frac{w_n + w_{n+1}}{2}\Delta p_n
\end{equation}
Es decir, entre cada par consecutivo de datos calculamos la razón de mezcla promedio y la multiplicamos por la diferencia de presión $\Delta p$

Realizando lo anterior entre los primeros 4 niveles (hasta aproximadamente 870 hPa), obtenemos una cantidad de 11.37 milímetros.

    \item Para obtener el volumen antes de reventar consideraremos que la ecuación de estado
\begin{equation}
    PV = nRT \label{estado}
\end{equation}
se cumple en el momento que fue soltado, y justo antes de reventar. Luego tenemos que
$$\frac{P_1V_1}{T_1} = \frac{P_2V_2}{T_2}$$
Como queremos el volumen final, despejamos $V_2$.
\begin{equation}
    V_2 = \frac{T_2}{T_1} \frac{P_1}{P_2} V_1 \label{V2}
\end{equation}
En el enunciado nos dicen que la temperatura dentro del globo siempre es igual que la temperatura del ambiente. Entonces $T_1$ y $T_2$ los podemos obtener de los datos.

Por otro lado nos dicen que la presión justo antes de reventar es igual a la exterior. Es decir, $P_2$ también lo podemos obtener de los datos, en particular será la menor presión registrada (punto de mayor altura).

Ahora, para obtener $P_1$ utilizaremos la ecuación de estado
\begin{equation}
    P_1 = \rho_1 R_\text{He} T_1 \label{e1}
\end{equation}
Notemos que usamos la constante de gases específica para el gas al interior del globo, el cual asumiremos que es Helio. Recordemos que la constante $R_\text{He}$ tiene el valor de $\frac{R}{M_\text{He}}$ con $R =8.314 \frac{\text{J}}{\text{K mol}}$ la constante de gases ideales $M_\text{He} = 4.003 \frac{\text{g}}{\text{mol}}$ la masa molar del Helio.\\

Remplazando \eqref{e1} en \eqref{V2} llegamos a
\begin{align*}
    V_2 &= \frac{T_2}{T_1} \frac{V_1}{P_2} \left( \rho_1 R_\text{He} T_1 \right) \\
        &= \frac{T_2}{P_2}m_1 R_\text{He} 
.\end{align*}
Remplazando los valores numéricos de $T_2$ (valor de temperatura cuando en el punto de menor presión; igual a 21.65$^\circ$C) y $P_2$ (registro de menor presión; igual a 93.55 hPa) obtenemos que el volumen alcanzado es aproximadamente de unos 26.17 m$^3$, alcanzando así un radio de unos 1.84 metros
    \item Consideraremos el primer y último registro que tenemos de la fase final. Estos tienen una presión de 93.55 hPa y 111.35 hPa, respectivamente. 

Si consideramos una atmósfera estándar, la altura en función de la presión viene dada por
\begin{equation}
    Z = 44307.7 \left( 1 - \left( \frac{P}{1013.25} \right)^{0.190284}  \right) \label{zpisa}
\end{equation}
con $P$ la presión en hPa. 

Evaluamos \eqref{zpisa} para ambos valores de presión y calculamos la diferencia de altura. Obtenemos un valor de 948.9 metros  



\end{enumerate}



\end{document}
